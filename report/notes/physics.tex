\section{Notes on the Physics}
\sectionframe

\begin{frame}{Quantities and Relations}
  \begin{columns}
	\column{0.5\textwidth}
		\begin{align*}
			B &= \mu_0 (H + M) \\
			\chi &= \pdv{M}{H} \\
		\end{align*}
	\column{.5\textwidth}
		\begin{align*}
			&\text{in Tesla:}\\
			&B: &&\text{Magnetic Flux Density}\\
			&H\;\mu_0: &&\text{external / vacuum Field}\\
			&M\;\mu_0: &&\text{magnetisation}\\
			\\
			&\text{unitless:}\\
			&\chi: &&\text{magnetic susceptibility}
		\end{align*}
  \end{columns}
\end{frame}

\subsection{Magnetic Ordering}

\begin{frame}{Magnetic Ordering / Phase}
	\centering
	How do we determine the magnetic ordering?\\
	For a given $H$ and $T$ using only the Kerr spectrum? -- Is it even possible?
	\\ \vspace{1em}
	\begin{itemize}
		\item Different peaks in the spectrum allows us to measure at different points in Brillouin Zone. $\Rightarrow$ possible?
	\end{itemize}
\end{frame}

\begin{frame}{Hysteresis loop for different ordering}
	\begin{columns}
		\column{0.7\textwidth}
		\fullgraphic{image8.png}
		\tiny
		Handbook of Magnetic Materials, Volume 23
		\column{.3\textwidth}
		What if the ordering changes with $H$?
	\end{columns}
\end{frame}

\begin{frame}{Methods and Particles}
	\begin{columns}
		\column{0.5\textwidth}
		Methods with corresponding probing energies:
		\begin{itemize}
			\item Optical Spectroscopy
			\item Raman Spectroscopy
			\item Neutron Spectroscopy
			
		\end{itemize}
		\column{.5\textwidth}
		Particles / Quasiparticles \\ with excitation / creation energies:
		\begin{itemize}
			\item inner e$^-$
			\item outer e$^-$
			\item valence e$^-$
			\item exciton
			\item magnon
		\end{itemize}
	\end{columns}
	\todo{Finish}
\end{frame}